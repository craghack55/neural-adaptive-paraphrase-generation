\documentclass[12pt,final,twoside]{report}
%%%%%%%%%%%%%%%%%%%%%%%%%%%%%%%%%%%%%%%%%%%%%%%%%%%%%%%%%%%%%
% Some credits:
% The template initially was created by Prof. Dr. Holger Karl/Uni Paderborn '2006
% and was expanded and updated by 
% - Stefan Heinrich/Uni Paderborn/Uni Hamburg since 2008,
% - Sven Magg/Uni Hamburg since 2013.
% Suggestions for changes are always welcome.
%%%%%%%%%%%%%%%%%%%%%%%%%%%%%%%%%%%%%%%%%%%%%%%%%%%%%%%%%%%%%
% Meta information:
\newcommand{\trtitle}{Adaptive Learning Strategies for Neural Paraphrase Generation}
\newcommand{\trtype}{Master Thesis} %{Bachelorthesis} %{Diplomarbeit} %{Dissertation}
\newcommand{\trcourseofstudies}{Intelligent Adaptive Systems} %{Bioinformatik} %{Intelligent Adaptive Systems} 
\newcommand{\trauthor}{Ethem Can Karaoguz}
\newcommand{\trauthordegree}{} %{\ B.Sc.}
\newcommand{\tremail}{4karaoguz@informatik.uni-hamburg.de}
\newcommand{\trmatrikelnummer}{6641214}
\newcommand{\trgutachterA}{\href{mailto:beimann@informatik.uni-hamburg.de}{Prof.Dr.Chris Beimann}}
\newcommand{\trgutachterB}{\href{mailto:yimam@informatik.uni-hamburg.de, milde@informatik.uni-hamburg.de}{Seid Muhie Yimam, Benjamin Milde}}
%\newcommand{\trbetreuung}{\href{mailto:tbd@informatik.uni-hamburg.de}{Dipl.-Inform. To Be Defined}}
\newcommand{\trfach}{Language Technology Lab}
\newcommand{\trdate}{22.08.2018}
\newcommand{\trkeywords}{Learning, Language, NLP, Deep Learning}
%Optionally: it is also allowed to provide a street name on the
%\newcommand{\trstrasse}{Streetname 42}
%\newcommand{\trort}{22527 Hamburg}

%%%%%%%%%%%%%%%%%%%%%%%%%%%%%%%%%%%%%%%%%%%%%%%%%%%%%%%%%%%%%
% Languages:

% Falls die Ausarbeitung in Deutsch erfolgt:
% \usepackage[german]{babel}
% \usepackage[T1]{fontenc}
% \usepackage[latin1]{inputenc}
% \usepackage[latin9]{inputenc}                     
% \selectlanguage{german}

% If the thesis is written in English:
\usepackage[english]{babel}                         
\selectlanguage{english}

%%%%%%%%%%%%%%%%%%%%%%%%%%%%%%%%%%%%%%%%%%%%%%%%%%%%%%%%%%%%%
% Bind packages:
\usepackage{acronym}                    % Acronyms
\usepackage{algorithmic}                % Algorithms and Pseudocode
\usepackage{algorithm}                  % Algorithms and Pseudocode
\usepackage{amsfonts}                   % AMS Math Packet (Fonts)
\usepackage{amsmath}                    % AMS Math Packet
\usepackage{amssymb}                    % Additional mathematical symbols
\usepackage{amsthm}
\usepackage{adjustbox}
\usepackage{booktabs}                   % Nicer tables
%\usepackage[font=small,labelfont=bf]{caption} % Numbered captions for figures
\usepackage{color}                      % Enables defining of colours via \definecolor
\definecolor{uhhRed}{RGB}{226,0,26}     % Official Uni Hamburg Red
\definecolor{uhhGrey}{RGB}{136,136,136} % Official Uni Hamburg Grey
\definecolor{uhhLightGrey}{RGB}{220, 220, 220}
\usepackage{fancybox}                   % Put equations in a frame
\usepackage{fancyhdr}                   % Packet for nicer headers
%\usepackage{fancyheadings}             % Nicer numbering of headlines
\usepackage[body={5.8in,9in}]{geometry} % Type area (size, margins...)

%\geometry{a4paper,outer=3.35cm}        % !!!Release version (Normal margins)
%\geometry{a4paper,outer=2.5cm}         % !!!Print version (Additional margin on the left for the binding)
%\geometry{a4paper}                     % !!!Proofread version (Additional margin on the right for corrections)
\geometry{a4paper,outer=3.134cm}        % !!!Draft version (Same margins on left and right)
%\geometry{paperheight=10.0in,paperwidth=6.4in,top=0.51in,left=0.3in}  % !!!Developer version (Minimal margins)

\usepackage{graphicx}                   % Inclusion of graphics
%\usepackage{latexsym}                  % Special symbols
\usepackage{longtable}                  % Allow tables over several parges
\usepackage{listings}                   % Nicer source code listings
\usepackage{multicol}                   % Content of a table over several columns
\usepackage{multirow}                   % Content of a table over several rows
\usepackage{rotating}                   % Alows to rotate text and objects
%\usepackage{siunitx}					% Nicer display of units and number according to SI standards
%\sisetup{tight-spacing=true}  
\usepackage[hang]{subfigure}            % Allows to use multiple (partial) figures in a fig
%HINT: subfigure may be depricated already - maybe try subfig instead 
%\usepackage[font=footnotesize,labelfont=rm]{subfig}    % Pictures in a floating environment
\usepackage{tabularx}                   % Tables with fixed width but variable rows
\usepackage{url,xspace,boxedminipage}   % Accurate display of URLs

%%%%%%%%%%%%%%%%%%%%%%%%%%%%%%%%%%%%%%%%%%%%%%%%%%%%%%%%%%%%%
% PDF Information und Definitions:
\author{\trauthor}

\ifx\pdftexversion\undefined
\usepackage{hyperref}
\else
\usepackage[colorlinks=false,           % link is colores (true) or has colored frame (false)
            linkcolor=uhhRed,           % case colorlinks=true: define color.
            urlcolor=uhhRed,
            citecolor=uhhRed,
            bookmarks,                  % Place bookmarks erstellen
            bookmarksopen=true,         % Bookmarks will be shown at start (true/false)
            pdfpagemode=UseOutlines,    
            bookmarksopenlevel=1,       % Define the depth of shown links
            bookmarksnumbered,          % Numbers of chapers in Bookmarks
            pdftitle={\trtitle},
            pdfsubject={\trtype},
            pdfkeywords={\trkeywords},
            pdfauthor={\trauthor},
            plainpages=false
            ]{hyperref}
\fi

\ifx\pdftexversion\undefined
\else
\pdfoutput=1                            % Disable PDF-Output
\pdfimageresolution=1200
\pdfcompresslevel=2                     % 0 = no compression, 9 = strongest compression
\fi

%%%%%%%%%%%%%%%%%%%%%%%%%%%%%%%%%%%%%%%%%%%%%%%%%%%%%%%%%%%%%
% Configurationen:

\hyphenation{whe-ther}                  % Manually use: "\-" in a word: Staats\-ver\-trag

%\lstloadlanguages{C}                   % Set the default language for listings
\DeclareGraphicsExtensions{.pdf,.svg,.jpg,.png,.eps} % first try pdf, then eps, png and jpg
\graphicspath{{./src/}}                 % Path to a folder where all pictures are located
\pagestyle{fancy}                       % Use nicer header and footer

% Redefine the environments for floating objects:
\setcounter{topnumber}{3}
\setcounter{bottomnumber}{2}
\setcounter{totalnumber}{4}
\renewcommand{\topfraction}{0.9}        %Standard: 0.7
\renewcommand{\bottomfraction}{0.5}     %Standard: 0.3
\renewcommand{\textfraction}{0.1}       %Standard: 0.2
\renewcommand{\floatpagefraction}{0.8}  %Standard: 0.5

% Tables with a nicer padding:
\renewcommand{\arraystretch}{1.2}

% Chapter and Sections will not be written in capitals
\renewcommand{\chaptermark}[1]{\markboth{\chaptername \ \thechapter.\ #1}{}}
\renewcommand{\sectionmark}[1]{\markright{\thesection.\ #1}}

% Avoid french spacing (double spacing after a full stop)
\frenchspacing

%%%%%%%%%%%%%%%%%%%%%%%%%%%%
% Additional 'theorem' and 'definition' blocks:
\theoremstyle{plain}
\newtheorem{theorem}{Theorem}[chapter]
%\newtheorem{theorem}{Satz}[chapter]    % Wenn in Deutsch geschrieben wird.
\newtheorem{axiom}{Axiom}[chapter]     
%\newtheorem{axiom}{Fakt}[chapter]      % Wenn in Deutsch geschrieben wird.
%Usage:%\begin{axiom}[optional description]%Main part%\end{fakt}

\theoremstyle{definition}
\newtheorem{definition}{Definition}[chapter]

%Additional types of axioms:
\newtheorem{lemma}[axiom]{Lemma}
\newtheorem{observation}[axiom]{Observation}

%Additional types of definitions:
\theoremstyle{remark}
%\newtheorem{remark}[definition]{Bemerkung} % Wenn in Deutsch geschrieben wird.
\newtheorem{remark}[definition]{Remark} 

%%%%%%%%%%%%%%%%%%%%%%%%%%%%
% Provides TODOs within the margin:
\newcommand{\TODO}[1]{\marginpar{\emph{\small{{\bf TODO: } #1}}}}

%%%%%%%%%%%%%%%%%%%%%%%%%%%%
% Abbreviations and mathematical symbols
\newcommand{\modd}{\text{ mod }}
\newcommand{\RS}{\mathbb{R}}
\newcommand{\NS}{\mathbb{N}}
\newcommand{\ZS}{\mathbb{Z}}
\newcommand{\dnormal}{\mathit{N}}
\newcommand{\duniform}{\mathit{U}}

\newcommand{\erdos}{Erd\H{o}s}
\newcommand{\renyi}{-R\'{e}nyi}
% it is recommented to define complex terms as expression/newcommand and use the expression in the tex instead.

%%%%%%%%%%%%%%%%%%%%%%%%%%%%%%%%%%%%%%%%%%%%%%%%%%%%%%%%%%%%%
% Document:

\begin{document}

\pagenumbering{Roman}                   % Roman pagenumbering for lists and meta pages
\renewcommand{\headheight}{14.5pt}      % Size of headings

\thispagestyle{empty}
\fancyhead[LO,RE]{}                     % Define the header style for the meta pages

%%%%%%%%%%%%%%%%%%%%%%%%%%%%
% Cover sheet

\begin{titlepage}
%---Possibility 1:
    \begin{flushleft}
        \includegraphics[width=67mm, scale =1.5]{uhhLogoL.pdf}\\
    \end{flushleft}
    \rule{\textwidth}{0.4pt}
        \newline
        \vspace{2.0cm}
        \begin{center}
          \LARGE \textbf{\trtitle}
        \end{center}
    \vspace{2.0cm}
    \begin{center}
      \textbf{\trtype}\\
      %im Arbeitsbereich \trfach\\
      at Research Group \trfach\\
      \trgutachterA\medskip\\
      %Fachbereich Informatik\\
      Department of Informatics\\
      %MIN-Fakult\"at\\
      MIN-Faculty\\
      Universit\"at Hamburg \\[1.0cm] %not allowed to translate Universit{\"a}t Hamburg
      %vorgelegt von \\
      submitted by \\
      \textbf{\href{mailto:\tremail}{\trauthor\trauthordegree}}\\
      %Studiengang:   \trcourseofstudies \\
      Course of study:   \trcourseofstudies \\
      Matrikelnr.:  \trmatrikelnummer \\
      %a\\
      on\\
      \trdate
    \end{center}
    \vspace{2cm}
    \begin{center}
    \begin{tabular}{ll}
    %Gutachter: & \trgutachterA \\
    Examiners: & \trgutachterA \\
                   & \trgutachterB \\
    %Betreuung: & \trbetreuung \\    	% Adviser are not allowed to demand getting mentioned here, but are happy getting credited by student's initiative
    \end{tabular}
    \end{center}
    \vfill
%    \begin{tabular}{l}
%    \trauthor \\
%    Matrikelnummer:  \trmatrikelnummer \\
%    \trstrasse \\
%    \trort
%    \end{tabular}
%    \newline
    \rule{\textwidth}{0.4pt}
    \newpage 
\end{titlepage}

    %backsite of cover sheet is empty!
\thispagestyle{empty}
\hspace{1cm}
\newpage

%%%%%%%%%%%%%%%%%%%%%%%%%%%%
% Abstract:
\section*{Abstract}\label{sec:abstract}

Continuously learning from a data stream in NLP applications, is an important challenge for deep neural models. This thesis studies the applicability and the behaviour of deep neural models in a setting where the training data comes from a continual data stream. We propose adaptive learning strategies for neural paraphrase generation which are based on incremental, transfer and active learning. We simulate a continuous data stream by dividing an existing dataset randomly into subsets and feeding these subsets to the model iteratively. We demonstrate that integration of a deep neural model with a continuous data stream, is possible. Experimental results show that the neural model is able to adapt to the continuous data stream, increasing its' performance through time. Additionally, one of the proposed incremental learning strategies is able to outperform traditional supervised learning, increasing the overall BLEU score by 2.76. Moreover the model is able to learn much faster with incremental learning than traditional supervised learning, former outperforming the latter with only 60 percent of the dataset. We make an analysis of transfer learning for paraphrase generation, to show if knowledge transfer is possible for paraphrase generation. We conduct experiments in order to determine what aspects of the neural network are actually transferred. Moreover we test four different transfer methods which are designed for specific use cases. The experimental results on transfer learning show that success of the knowledge transfer depends on the context and size of the source and target datasets. We show that when the transfer is successful, both hidden layers and word embeddings are transferred. The results regarding the different transfer methods show that restricting the model on target dataset leads to poor performance especially in incremental setting. Sampling methods we used for active learning are proven to be ineffective on paraphrase generation task leading no improvements on both overall BLEU score and training data usage. We also show that in case of a concept drift, the model is able to successfully adapt to the new training data despite the fact that it is previously trained on a different dataset.


%\fancyhead[LE,RO]{\it Abstract}
%\addcontentsline{toc}{chapter}{\numberline{}Abstract}
	


%% If the abstract ist not longer than half a page, then the German Zusammenfassung can be places on the same page
%%\cleardoublepage
%\vfill
%\section*{Zusammenfassung}\label{sec:zusammenfassung}
%%\addcontentsline{toc}{chapter}{\numberline{}Zusammenfassung} %Add the Zusammenfassung to the TOC
%Hier die deutsche Zusammenfassung einf\"ugen (notwendig).
%\fancyhead[LE,RO]{\it Abstract}

%\cleardoublepage

%%%%%%%%%%%%%%%%%%%%%%%%%%%%
% Lists:
%\setcounter{tocdepth}{1}               % depth of the table of contents (for BSc and MSc Thesis 1 is recommented)
\fancyhead[LE,RO]{\it Contents}
\tableofcontents
\cleardoublepage
% List of Figures and List of tables are optional. -> Not needed in most theses.
\fancyhead[LE,RO]{\it List of Figures}
\listoffigures
%\cleardoublepage
\fancyhead[LE,RO]{\it List of Tables}
\listoftables
%\cleardoublepage
%\lstlistoflistings
%\cleardoublepage

\fancyhead[LE]{\it \leftmark}           % Define the header style for the text pages
\fancyhead[RO]{\it \rightmark}          % Define the header style for the text pages
\fancyhead[LO,RE]{}                     % Define the header style for the text pages

%%%%%%%%%%%%%%%%%%%%%%%%%%%%
% The content will be included here:
\pagenumbering{arabic}

\chapter{Introduction}\label{introduction}

Generally, machine learning applications build models depending on training data which is gathered collectively by using a specific data acquisition tool. This process introduces a number of problems regardless of quality and amount of the gathered training data. First of all, the data acquisition process itself can cost lots of time and money since the data has to be labelled by human workers. Moreover if the domain in which training data is collected from is very specific for example, medical text, certain number of domain experts might be needed. In the cases where the domain experts can not be utilised, gathering necessary amount of data to build a model can be impractical or impossible. Secondly, training on previously gathered large datasets with traditional supervised learning can be very computationally expensive. Thirdly and most importantly, data distribution of the domain where target application operates can change which leads to existing training data to become outdated. Since the model is not trained with changed data coming from the environment, it performs poorly. This phenomenon is called concept drift in the literature \cite{concept}. In this case more training data has to be gathered and the model has to be trained again with new data both of which can be again impractical or impossible depending on the resources at hand.

These mentioned problems make us to think about a data acquisition process where training data is collected interactively and iteratively with a model which adapts to the continuous data stream. In the proposed design, the model is trained continuously with each iteration of data acquisition process instead of training by traditional supervised learning with large amount of data. The model not only aims to automatically adapt to new training data without any retraining from scratch, it also interacts with human users by providing suggestions to users and making use of user feedback as a form of input to train. With this human-in-the-loop paradigm, necessity of collecting a large dataset in advance is eliminated and a functional model (even though it is not fully optimal to use) is created very fast, at the same time. We call this paradigm adaptive learning. Figure 1.1 shows the basic workflow of suggested data acquisition process. We aim to create a model which has comparable or better performance than a model trained with traditional supervised learning at the end of data acquisition process. Figure 1.2 shows a real world human-in-the-loop data acquisition system for text simplification.

\begin{figure}[h]
\includegraphics[width=\textwidth, height=10cm]{hil}
\centering
\caption{Human in the loop data acquisition}
\end{figure}

Main focus of this thesis is the behaviour of deep neural models in human-in-the-loop settings which is described above. We mainly study different learning strategies for training deep neural models in this specific setting, conduct experiments with different training methods to investigate their effectiveness in continuous data streams. We also propose and explore additional learning strategies to overcome some of the problems which are introduced by human-in-the-loop setting. Although human-in-the-loop data acquisition can help with the problems which are described in the beginning of this section, it also introduces some other problems regarding in the context of modelling. We start with proving that a deep neural model in fact can adapt and improve with the continuously increasing data and move to experiment with different learning strategies each of which addressing a different problem in human-in-the-loop setting. We study incremental learning instead of traditional supervised learning, in order to obtain adaptivity to incoming data without forgetting previous knowledge. We experiment with different ways to train the model continuously mainly related to how the model processes incoming data. We also show that with incremental learning we can create a relatively good and usable model, way before we are exposed to all of the training data. We experiment with transfer learning in order to simulate a concept drift scenario where a general model comes across to new training data with different nature. We also simultaneously simulate a situation in which we do not have enough resources to collect significant amount of data. We show that with transfer learning, it is possible to create model which is impossible to create with available data. Additionally we study different active learning strategies in order to reduce the data we need to build a satisfactory model. At the end we also do experiments in which we combine these strategies together and report the results.

\begin{figure}[h]
\includegraphics[width=\textwidth]{human-in-the-loop}
\centering
\caption{An interactive human-in-the-loop application for text simplification \cite{par4sim}. Target words to be simplified are highlighted and user is provided with a list of candidate words to choose from. Candidate words are generated and ranked by the system/model.}
\end{figure}


As the target application, we choose paraphrase generation. Paraphrase generation is the problem of generating different texts from a source text while retaining the meaning. It has various application areas in Natural Language Processing (NLP) like dialogue systems where it is used for building more natural conversional agents, information retrieval where it is used for enhancing retrieval process, natural language generation where it is used for generating training data for different NLP tasks and text summarization where it is used for replacing a text with a simpler paraphrase of it. Generating the text "President Trump strongly denied any wrong doing in this matter." from the text "POTUS firmly denied any fault in this subject." would be a good example of paraphrase generation. It is a challenging problem because of many different reasons. There are more than one way to paraphrase a source sentence and the quality of generated paraphrase depends on many things like context information, lexical and semantic diversity and so on. Therefore the problem is not only concerned with language generation, it also deals with language understanding. Another problem which is introduced by picking paraphrase generation as our target application is out of vocabulary words. Deep neural models for NLP problems is built on predefined and specific vocabularies which are usually constructed from training data. These vocabularies cannot be changed or expanded during or after training. Since the model does not know the words which it have not seen in the training, unknown words can lead to poor performance. This is an open problem in NLP and all neural models suffer from it but in our case it is more problematic because of the setting we study and nature of the problem we deal with. In our setting data comes to the system continuously and we train the model continuously therefore adapting model strictly depends on the vocabulary which it started training with. Additionally paraphrase generation is not a classification or regression task, it is a generation task therefore we do not know target texts (paraphrases) for our source texts beforehand, they are generated by users. These two mentioned reasons cause our model to be more prone to out of vocabulary words since there will be more of them compared to traditional supervised setting. There are two possible solutions for this problem. First solution is to use pre-trained word embeddings for our neural net instead of training our own word embeddings. These pre-trained word embeddings are built on very large corpuses so they would cover a lot of out of vocabulary words for our training data. Second solution is to design a data acquisition process similar to \cite{par4sim} where the system suggests paraphrase candidates which are created from large, existing resources and make the user choose from suggestions. Even though this puts restrictions to the users, it ensures that the target paraphrases are built from a known vocabulary. Since we know the vocabulary beforehand we could train our own word embeddings. In this thesis we simulate the latter solution. 

We simulate the human-in-the-loop described in this section by dividing a large existing dataset into subsets and feed the model with these subsets iteratively. In each iteration, the model gets a new batch of data and updates itself according to it. We train the model with these batches using different learning strategies and evaluate it on a separate test set at the end of each iteration, observing the model's behaviour through time.

According to our knowledge there is no work studying neural paraphrase generation in human-in-the-loop settings especially concerning adaptivity of the neural model. Moreover this work seems to be first to try incremental learning and active learning for neural paraphrase generation. We also build on to existing transfer learning research for neural paraphrase generation by experimenting with different methods of transfer and analysing the transfer process.

\section{Thesis Organization}
This thesis is organized as follows; the rest of this chapter describes the research question we would like to investigate and review state-of-the-art works in neural paraphrase generation, human-in-the-loop learning and the strategies we use in our work. In Chapter 2, we give the main methodology and technical details regarding our deep neural model and main ideas behind the strategies we use. Chapter 3 describes the details of our learning strategies, the datasets and evaluation metric we use. It also explains the experimental setups we have in detail with their corresponding reasonings. In Chapter 4 we present our results of the experiments and explain the findings. The last chapter gives a discussion on the overall findings of thesis, an extensive list of future work and the conclusion.

\section{Research Questions}

This thesis is going to aim to investigate whether deep neural models can be combined with adaptive learning, human-in-the-loop paradigm for paraphrase generation task. The basic research questions can be summarized as below:

\begin{itemize}
  \item Can deep neural models effectively adapt and gradually perform in a continuous data stream for paraphrase generation?

Considering the fact the deep neural models need a lot of training data to perform well, can we integrate this type of models into human-in-the-loop setting? We would like our model to learn and adapt continuously since we do not want to retrain large models over and over again through time. We would like to see that if such models can be applied to real world applications which have similar setting to ours. If the answer is yes, what should be the training scheme (how to make use of incoming data stream) for such models? Basically we would like to know if continuous learning is possible for neural paraphrase generation. 
 
  \item Can we achieve better or comparable performance than traditional supervised learning by leveraging the data stream?
  
  Can adaptive learning improve accuracy by adding more generalization power? Can the model achieve better or comparable performance with less data? It is also possible that training a deep neural network in adaptive manner may have a negative effect, if so it will be insightful to understand the reason why. Understanding how deep models behave under this training paradigm is an interesting and challenging question.
   
  \item What are the possible challenges/limitations introduced by system and model in this setting and what are the possible learning strategies we can use for dealing with these challenges?
  
  We would like to know what kind of existing learning strategies we can use to perform better in human-in-the-loop setting. This is particularly important since the setting itself imposes some challenges. How can we deal with challenges like limited resources and concept drift? Additionally training deep neural models continuously introduces challenges of its own like overfitting and underfitting depending on the training data.

\end{itemize}

Considering the lack of research in this area, with these questions answered, it might be possible to enhance the performance of deep neural models (shorter training time and / or better paraphrases) in paraphrase generation task and create a framework on adaptive learning for future research.

\section{Related Work}

Deep learning based approaches have been quite successful in various NLP tasks including language modelling \cite{siriam}, automatic speech recognition \cite{hannun} and neural machine translation \cite{cho}. Main idea behind deep learning is learning future representations of the dataset instead of depending on hand crafted features engineered by domain experts. Deep neural models is capable of hidden representations and relationships inside the data which are not possible to obtain by feature engineering. In the case of NLP, deep neural models are capable of capturing lexical, semantical and contextual relationships between textual data points which makes them quite successful. Nowadays, almost all of the state-of-the-art models used in NLP applications are based on deep learning.

For the task of paraphrase generation deep learning is already proved to be extremely successful by the work which has been done in past few years. Especially with the development of sequence to sequence learning \cite{Vinyalsetal}, a lot of research has been emerged, building models for paraphrase generation using this framework. \cite{Prakashetal} introduces a sequence to sequence model based on stacked LSTM's with residual connections. The authors use residual connections between stacked LSTM networks in order to address exploding/vanishing gradient problems for LSTM based models. They have experimented on different large scale datasets to show the effectiveness of their model which can produce meaningful, semantically and grammatically correct paraphrases. They also show that with residual connections, the LSTM based model can learn how to retain important words better. We use a lightweight variant of this model in this thesis.

\cite{Guptaetal} uses deep generative models coupled with LSTM's in sequence to sequence framework, in order to generate paraphrases. They modify the model by conditioning generative model with source sentence in order to enhance performance. They have managed to outperform state-of-the-art methods for paraphrase generation and generate a baseline for Quora Question Pair dataset which we use in this thesis. Additionally they evaluate their model with human evaluation which suggest the model can create good and relevant paraphrases. This work is particularly interesting since the resulting model can also link unseen concepts which are related to the original sentences.

\cite{Lietal} uses a very interesting approach, using recently popularized deep reinforcement learning for paraphrase generation. Their approach uses two components, a generator which is a sequence to sequence model, responsible for creating paraphrases, and an evaluator which is a deep matching model, responsible for recognizing paraphrases. Main idea is the evaluator enhancing the performance of generator. They propose this framework as a solution for lack of evaluation metrics for paraphrase generation. Their approach outperforms the state-of-the-art methods in paraphrase generation in both human evaluation and automatic evaluation.

Interactive data acquisition tools for NLP tasks has been worked on quite extensively in recent years. Many different tools are designed and implemented in order to enhance the ability to collect and use complex text data. They make use of multiple visualizations, correction mechanisms and machine learning based feedback loops. \cite{trivedi} developed an interactive user interface for enabling NLP tasks on clinical text which requires domain expertise. \cite{Yimam:2016aa} developed a tool for applying interactive, human-in-the-loop machine learning to biomedical entity and relation recognition. They demonstrate the effectiveness of their approach by simulating human-in-the-loop data acquisition with three experiments and prove their concept for adaptive learning. Following this work, \cite{par4sim} creates an adaptive learning system for text simplification which is integrated with a machine learning model. The system effectively makes use of the model in the data acquisition process, creating a feedback loop with human workers. They have done a real-time data acquisition using the Amazon MTurk crowdsourcing platform. They showed successful integration of their model and its adaptiveness through time by evaluating the model's performance and showing its improvement over time. This thesis is inspired by this particular work, replaces the model used with a deep neural model and studies its behaviour. For the adaptive, continual learning in neural networks \cite{parisi} gives an extensive review on the subject, covering different number of techniques (some of them are utilized in this thesis) to achieve adaptive learning. In this thesis, we study adaptive learning in the context of both applicability in real world and learning capability.

In case of transfer learning, it has been already established that knowledge transfer is possible and quite effective in computer vision. Recently, there has been substantial work on its possible applicability on the field of NLP also. An extensive look on knowledge transfer on different NLP tasks is provided in \cite{mou}. They do systematic experiments on different NLP problems to determine what kind of aspects of natural language and network layers are transferable. They show that knowledge transfer is possible between different datasets of same task and semantically similar tasks even though improvement from transfer learning depends on the datasets and the tasks at hand. \cite{zoph} successfully applies transfer learning to neural machine translation, gaining significant increase in performance for translation of low-resource languages. \cite{brad} uses transfer learning for neural paraphrase generation, effectively boosting the performance. Even though they successfully show that knowledge transfer is possible for neural paraphrase generation they do not study different ways to transfer. In our thesis, we conduct a layer-by-layer analysis for studying different transfer methods and what part of the network is being transferred. Finally, \cite{yoon} studies different transfer schemes for creating personalized language model from a general model. Building on these previous works, we provide a detailed investigation of transfer learning in neural paraphrase generation.

For active learning, as far as our knowledge is concerned there is no previous work. \cite{shen} successfully applies active learning for name entity recognition based on different sampling strategies. \cite{zhao} uses a sampling strategy based on a combination of model uncertainty and representativeness of data points, applying it on short-text classification. \cite{rubio} successfully applies it on interactive machine translation. Even though it has never been applied on paraphrase generation, these works show that active learning can be used for a variety of NLP tasks. We use some of the sampling strategies used in these works and try to apply them to paraphrase generation.

\chapter{Methods}\label{methods}

This chapter provides necessary background knowledge about the fundamental methods we use in this thesis. First we briefly explain the main ideas behind the deep neural model we use for paraphrase generation. After that we explain the learning strategies/approaches we use in our research.

\section{Recurrent Neural Networks}

\begin{figure}[t]
\includegraphics[width=\textwidth]{rnn}
\centering
\caption{Recurrent neural network expanding through time steps \cite{zhao}}
\end{figure}

Recurrent neural networks (RNNs) are a type of neural network model designed for processing sequential data. The data is sequential which means it is processed in time steps by the model. This notion of time step does not have to mean literal concept of time we have in real life. For example, it can be the position in the sequence (which is exactly the case for NLP). Since the length of a sequence can be really long, the model uses parameter sharing instead of optimizing separate parameters for each time step. This parameter sharing paradigm helps model to generalize to different sequences by learning relationships between different time steps and different sequence lengths. Figure 2.1 shows a simple RNN and its expansion with respect to time. As it can be seen from the figure, the network produces an output in every time step.

Recurrent neural networks have numerous variants usually constructed by using different ways of connecting network layers or introducing additional computational mechanisms in order to increase generalization capability. 

\section{Long Short-Term Memory}

\begin{figure}[t]
\includegraphics[scale=10]{lstm}
\centering
\caption{A typical LSTM cell \cite{paszke}}
\end{figure}

Long Short-Term Memory (LSTM) is a variant of RNN that adds extra computational mechanisms to the network in order to avoid the vanishing and exploding gradient problems. These computational mechanisms includes a memory cell and a set of logical gates which helps the model to learn memory required tasks. Especially in NLP this is an important aspect since natural language contains context relationships between words which can be observed at different time steps. 

Specifically LSTM adds a memory cell $c_{t}$ for every time step t. At each time step t, a single unit works with the input state $x_{t}$, the hidden state $h_{t-1}$ and the memory state $c_{t-1}$ to calculate the hidden state $h_{t}$ and the memory state $c_{t}$. The memory cell has three computational mechanisms: input gate i, forget gate f, and output o. These gates are also trained meaning that their weights are updates with gradient descent with respect to time. It is known that learning these gates helps with gradient explosion.

Figure 2.3 shows a basic LSTM cell. Equations for calculating the elements of an LSTM cell are \cite{paszke}:

\begin{equation}
i_{t} = \sigma(W_{xi}xt + W_{hi}h_{t-1} + b_{i})
\end{equation}

\begin{equation}
f_{t} = \sigma(W_{xf}xt + W_{hf}h_{t-1} + b_{f})
\end{equation}

\begin{equation}
o_{t} = \sigma(W_{xo}xt + W_{ho}h_{t-1} + b_{o})
\end{equation}

\begin{equation}
c_in_{t} = \tan(W_{xi}xt + W_{hi}h_{t-1} + b_{i})
\end{equation}

\begin{equation}
c_{t} = f_{t} \odot c_{t-1} + i_{t} \odot c_in_{t}
\end{equation}

\begin{equation}
h_{t} = o_{t} \odot \tan(c_{t}) 
\end{equation}

Parameters of the above equations are:

\begin{itemize}

\item $W_{x\_}$ and $W_{h\_}$: Weights for input x and hidden state h respectively.

\item $\sigma(\_)$ and $\tan(\_)$: Element-wise sigmoid function and hyperbolic tangent function.

\item $\odot$: Element-wise multiplication.

\item b: Bias parameter.

\end{itemize}

\section{Sequence to Sequence Model Architecture}

First introduced by \cite{Vinyalsetal} neural sequence to sequence framework consists of two components, encoder and decoder. Encoder creates a low dimensional representation of the source sequence called thought vector. It aims to capture semantical and contextual relationships of the source sequence. Thought vector then is fed into decoder which produces a high dimensional target sequence. The process is shown in Figure 2.1. Main idea of the encoder-decoder pair is creating a mapping between words and vectors, in other words capturing the meaning as numerical vectors. By design, both encoder and decoder should be a recurrent neural network or a variant of it. In our model, we use stacked LSTM's for both encoder and decoder. During the decoding process, generation of new words depends on the word generated before by decoder. The decoder starts generating the target sequence when it recognizes the 'EOS' (end-of-sentence) character in the source sequence. EOS character is usually appended to the end of source sequence. The model maximizes the log probability of the target sequence given the source sequence.

\begin{figure}[t]
\includegraphics[scale=0.45]{seq2seq}
\centering
\caption{Sequence to sequence learning with encoder-decoder framework}
\end{figure}

\section{Incremental Learning}

Incremental learning is a learning scheme based on training the model continuously in order to adapt to new training data without losing existing knowledge base. It is mainly used in cases where there is a data stream, constantly providing new data points. Incremental learning based models are mainly used to tackle problems like data availability and low resources. Incremental learning becomes essential in human-in-the-loop settings where the model has a feedback loop with users. Main concern of incremental learning is adaptivity since the model is especially prone to changes in training data distribution, a situation called concept drift which is explained in previous section. 

Additionally the specification of how to adapt the model creates another problem. Depending on the task and data at hand, exactly how the model should deal with the balance between old and new information has to be decided beforehand. Many incremental learning strategies for deep neural models consist of design decisions (hyper-parameters, network topology etc.) in order to deal with this problem. For example, using a large learning rate can lead the model to quickly overwrite existing knowledge with most recent data points. In our work we experiment with some of these strategies on paraphrase generation.

\begin{figure}[t]
\includegraphics[width=\textwidth, height=10cm]{transfer-learning}
\centering
\caption{Idea behind transfer learning}
\end{figure}

\section{Transfer Learning}

Transfer learning is based on the idea of using the knowledge gained from previously learned tasks and datasets. Usually it is used for decreasing the amount of training data and time of a low resource task where obtaining more training data is impractical. Moreover it is used for enhancing the recognition capabilities of an existing model, for example adding a new label for classification image recognition. This is a very important use case especially if the existing model is large since transferring knowledge eliminates the need of retraining the whole model. It can be also used for personalize an existing large model with a new domain specific dataset, enhancing its performance on new domain. Figure 2.1 shows the basic workflow of transfer learning.

Deep neural models learns multiple hidden representations of the datasets, some part of which are shared between different tasks and datasets. Therefore transfer learning for deep neural models usually consists of copying layer weights across models and applying restrictions to new model in order to preserve existing knowledge. In the field of NLP, this process includes transferring learned word embeddings (numerical representations of words) and hidden layers (learned relationships of the datasets like context). Depending on the target task and dataset, the method of transfer which basically represents how we train the target model, can significantly change. Different methods of transfer can include changing hyper-parameters, freezing network layer or adding new layers.

\section{Active Learning}

Active learning is a learning scheme based on selecting training samples to be annotated by using intelligent strategies. These strategies employ certain criterions in order to create an opinion on how informative or useful a data point is and they are called sampling methods. Informativeness of a data point can depend on dataset, task and model therefore selection of the sampling method is very crucial. The criterions considered by a certain sampling method could simply be a heuristic about the nature of dataset at hand or they can depend on the opinion of model about the data points or they can depend on feature representations of the dataset like similarity measures. Usually, sampling methods combine multiple criterions for evaluating the informativeness of a data point and some of these criterions may require complex algorithms to be obtained. 

\begin{figure}[t]
\includegraphics[width=\textwidth, height=10cm]{active-learning}
\centering
\caption{Pool based active learning scheme}
\end{figure}

Figure 2.2 shows the basic workflow of active learning scheme. Main purpose of active learning is obtaining greater or similar performance with fewer labeled training data since collecting a large, labelled training data can be difficult, costing time and money. In our work, we use active learning as an option for enhancing human-in-the-loop data acquisition by asking the users to annotate most difficult samples. Additionally we would like to see if we can make the model more adaptive by using active learning. It is also worth studying what constitutes an informative sample when it comes to paraphrase pairs.

\chapter{Adaptive Learning Strategies}\label{approach}

This chapter explains our approach for adaptive neural paraphrase generation. It describes the main learning strategies that are used in detail including what purpose do they serve in our human-in-the-loop setting. First, details of the model which are used in the experiments are explained. Then descriptions of the datasets and the evaluation metric are presented. For last, the learning strategies and the experimental setups are discussed.

\section{Model Details}

In this work, a lightweight variant of the model proposed in  \cite{Prakashetal} with bahdanau attention \cite{bahdanau} is used. The reason the exact same model from the paper is not used is to fully explore all learning strategies since even with a lightweight version of the model, training takes a long time. The model keeps the same stacked LSTM based multi-layer architecture with residual connections but using 3 layers instead of 4. Figure 3.1 shows the basic unit in the model.

\begin{figure}[t]
\includegraphics[width=\textwidth]{residualLSTM}
\centering
\caption{Stacked LSTM based model \cite{Prakashetal}.}
\end{figure}

During the experiments, the model has 512 units in each LSTM layer. The encoder and decoder use a maximum length of 100 for input and output both. The model trains its own word embeddings with the embedding size of 512. In all experiments, the model is trained with a batch size of 50 and an exponential learning rate decay with the square root function.

In the cases when it improves the performance, a beam search with beam size of 5 is used.

\section{Datasets and Evaluation Metric}

Four different datasets are used in the experiments, varying in context and complexity. 

The Microsoft Research Paraphrase Corpuus (MSR) \cite{msrp} is a small but very challenging dataset which contains 5800 pairs of paraphrases extracted from news sources. Since the paraphrase pairs are extracted from different news sources, the dataset contains a lot of lexical and contextual variety. It also contains a lot of special words and concepts. Because of these reasons, MSR corpus is a very hard dataset to create paraphrases from even with the state-of-the-art models with high computational resources. The dataset is separated into subsets of 2753, 997 and 150 for training, test and validation respectively.

The Quora Question Pairs (QUORA) is a dataset consisting of question pairs which are labeled as paraphrases. It is originally a dataset for paraphrase recognition not paraphrase generation. The positive class of dataset is filtered and 155,000 paraphrase pairs are collected. There are one to many relationships in the dataset which means it contains multiple paraphrases for same source text. This property makes the dataset an ideal candidate for paraphrase generation. The dataset is separated into subsets of 119445, 22861 and 2000 for training, test and validation respectively.

The Microsoft Common Objects in Context (MSCOCO) \cite{mscoco} is a large dataset consisting of human generated image captions with 351,163 pairs. It also has one to many relationships. MSCOCO is one of the most popular datasets used in paraphrase generation literature. All of the dataset is used for training.

The PPDB Lexical (PPDB) \cite{ppdb} is a large dataset which contains 500,000 pairs of short paraphrases and it is also very popular in paraphrasing research. The paraphrases in this dataset are short that lack context information. All of the dataset is used for training.

\begin{table}
\small
 \begin{tabular}{||c c c c||} 
 \hline
 Label & Sentence & Dataset & \\ [0.5ex] 
 \hline
 source & the dvdcca then appealed to the state supreme court & MSR & \\
 \hline
 target & the dvd cca appealed that decision to the us supreme court & MSR & \\
  \hline
 source & why do rockets look white & QUORA & \\
 \hline
 target & why are rockets and boosters painted white & QUORA & \\
 \hline
 source & a blue and white bathroom with butterfly themed wall tiles & MSCOCO & \\
 \hline
 target & an angled view of a beautifully decorated bathroom & MSCOCO & \\
 \hline
 source & despicable & PPDB & \\
 \hline
 target & contemptible & PPDB & \\
 \hline
\end{tabular}
\caption{Example paraphrase pairs from the datasets.}
\end{table}


Table 3.1 shows example paraphrases from all the datasets used in the experiments. As it can be seen from the table, the datasets are varying in paraphrase complexity therefore it is easy to see why MSR dataset is harder to paraphrase than the others. The MSR dataset is used as a target dataset on which the model is built and evaluated, whereas MSCOCO and PPDB are used as source datasets for transfer learning. The QUORA dataset is used as a source and a target dataset in different experimental setups.

There are no evaluation metrics especially designed for paraphrase generation. Therefore for evaluation the metric BLEU \cite{Papinenietal}  is used. BLEU is a widely used metric for paraphrase generation even though it is designed for machine translation. It is seen as an acceptable option for paraphrase generation since it is shown that the score correlates well with human evaluation.

\section{Experimental Setups}

The human-in-the-loop data acquisition process is simulated by randomly dividing a dataset into train, test and validation, taking training set's subsets and feeding it to the model in each iteration. After each iteration the model is trained with the data available according to learning strategy which is experimented on and it is evaluated iteratively with the test set. The test set is preferred to be relatively large in order to test how well the model generalize after each iteration. The model is trained with same amount of epochs and it starts the training with same hyperparameter configuration through the iterations in order to compare the performance without bias. If it is necessary, parameter tuning is done on the validation set beforehand and the best model is taken for the experiments with learning strategies. Every learning strategy is experimented with the same model (same architecture, same hyperparameters etc.).

\section{Incremental Learning Strategies} \label{incs}

In the continuous setting this thesis studies, the training data is limited at the beginning. After a certain amount of time, the system is able to collect enough data to build a stable model with supervised learning. This model then, can be used for application purposes as long as it performs at an acceptable level. There are two major drawbacks in this solution. Firstly, the model is not able to use the training data which is collected after its creation. This is clearly a waste of resources since the training data comes from a continuous stream. Secondly and most importantly, the model would have to be trained on a regular basis just to keep up with the new training data. This is not desirable and not practical, even impossible depending on the circumstances. Therefore, incremental learning is a natural option in this case.

The main concern in human-in-the-loop setting is to make continual, data-driven learning possible since traditional supervised learning is not practical. The model should keep learning with incoming training data which is provided by the data stream, adapt itself to the changes in dataset without forgetting the knowledge it learned before. Since the model is getting training data in small chunks, one of the most important design decisions which has to made is how to process available training data through time. In the case of neural models, learning basically means to update model's weights by iterations according to the training data. Because of the very nature of gradient descent based learning the more certain data points are used in training the more the model is going to adjust itself towards to those data points. This could lead to a bias towards the training data which is gathered at past, severely hurting the model's learning capability. It is important to notice that this can also happen within the same dataset if the dataset has high variance. This problem would not occur if the model is trained with supervised learning since the whole dataset would be available for the training.

As explained in previous sections, concept drift is the other big challenge that has to be considered when learning in continuous data streams. Optimally model should be able to pick up the statistical and conceptual changes in the incoming training data and update its weights accordingly. The model should be able to do this without forgetting previously learned knowledge so that it can preserve its original functionality. After adapting to the new training data, it should still be able to perform reasonably well when it encounters test data from the original dataset. 

It is reasonable to say that when the two problems described above are considered, adaptivity of a model is basically a tradeoff between how fast it learns the new knowledge and how fast it forgets the old one. Any learning strategy that is to be used in adaptive learning, should employ some measures in order to preserve the balance between learning and forgetting. As it is said before because of the very nature of how deep neural models learn, it is impossible to avoid either one of them but it is possible control the rate of how fast the model learns and forgets.

Two different incremental learning strategies which differ in how they process the incoming training data, are proposed and experimented with. The model is regulated with different methods available in traditional supervised learning. The behaviour of these two incremental learning approaches are studied and some insights on adaptive paraphrase generation, are provided. The training and regularization methods that are used, are fairly simple but they haven't been studied before so the finding of this thesis is thought as a foundation for more complex approaches for adaptive learning in NLP.

\subsection{Incremental Learning with Data Pooling (IL1)}

The first learning strategy trains continuously (weights are not re-initialized between iterations, the model uses and updates the same weights) from a data pool which contains all the data collected in previous iterations. The model uses all the data available to it regardless how old or new the data is. Figure 3.2 shows the basic work flow of the learning strategy. As it can be seen from the figure at the end of each iteration, the collected data is placed in a data pool and the model trains itself using the updated data pool. After the training process, next iteration starts and cycle continues. The purpose of this learning strategy is mainly show that if the neural model can constantly improve itself through time to the incoming data. Main focus is not to forget previously attained knowledge while learning new training data reasonably well and make model to use all the resources available. The advantages of this learning strategy can be listed as:

\begin{figure}[t]
\includegraphics[width=\textwidth]{IL1}
\centering
\caption{Incremental learning with data pooling (IL1). The model uses data from all previous iterations, hyperparameters re-initialized and weights are transferred through iterations.}
\end{figure}

\begin{itemize}

  \item Since the model trains on all data available at the end of each iteration, it is unlikely to forget learned knowledge from previous iterations.
  \item The model has the chance to learn more difficult data points better especially if they are encountered in early iterations. Basically instead of waiting to get enough data to learn with supervised learning, that time can be used to work on challenging data points seen by the model.

\end{itemize}

The disadvantages of this learning strategy are:

\begin{itemize}

  \item Since the size of data pool is increasing linearly, the time model takes to train at the end of each iteration and the required memory for data pool, increases as well. This increase can be linear or exponential depending on the model.
  \item The model sees data points from earlier iterations way more often which means it updates its weights according to those data points more often. This can cause overfitting to those particular data points which can diminish the performance of the model.

\end{itemize}

\subsection{Incremental Learning without Data Pooling (IL2)}

The second learning strategy trains continuously (weights are not re-initialized between iterations, the model uses and updates the same weights) only from the data of last iteration. There is no data pool that aggregates collected training data. Figure 3.3 shows the basic work flow of the learning strategy. As it can be seen from the figure, the model only updates itself with the most recent iteration. After the training process, next iteration starts and cycle continues. Purpose of this learning strategy is to create an efficient model in terms of the training time and memory. The main focus is the adaptivity of model to new data. The advantages of this learning strategy can be listed as:

\begin{figure}[t]
\includegraphics[width=\textwidth]{IL2}
\centering
\caption{Incremental learning without data pooling (IL2). The model only uses data from latest iteration, hyperparameters re-initialized and weights are transferred through iterations.}
\end{figure}


\begin{itemize}

\item Since the size of data pool is constant, it is efficient in terms of time and memory.
\item The model sees all data points for equal amount of times therefore it does not have the problem of overfitting as incremental learning with data pooling.

\end{itemize}


The disadvantages of this learning strategy are:

\begin{itemize}

\item Since the model trains on iterations separately it is prone to overwrite and forget old knowledge from previous iterations.
\item All training points are updated for same amount of iterations. If the number of epochs are not large enough the model can underfit and fail to generalize.

\end{itemize}

\subsection{Incremental Learning with Network Expansion (IL-NE)}

The last learning strategy trains continuously (weights are not re-initialized between iterations, model uses and updates the same weights) and expands the neural network after certain number of iterations. In this context network expansion means adding another LSTM layer to the network. Purpose of this learning strategy is to increase adaptive capabilities of the model. Main focus is to deal with possible concept drift which can occur during data acquisition. By adding randomly initialized new layers when more than a certain amount of data is introduced to the model, specialized layers focusing on the old and new information are aimed to be created. Experiments are conducted with expansion in every third iteration. Different experiments are done with and without freezing the first active layer in every expansion step in order to study if restricting the model through time helps with adaptivity.   

\subsection{Incremental Learning Baseline (IL3)}

The model is also trained from ground zero (with the weights re-initialized) from the data pool at the end of every iteration in order to create a comparable baseline. The baseline model which is trained after the last iteration is basically the model trained with supervised learning since it uses the whole training data available. The basic idea is to check if the incremental learning strategies achieve better or comparable performances than last baseline model. Figure 3.4 shows how the baseline is created.

\begin{figure}[t]
\includegraphics[width=\textwidth]{IL3}
\centering
\caption{Incremental learning baseline (IL3). The model is trained from scratch for each iteration, no knowledge transfer involved.}
\end{figure}

\subsection{Model Regularization in Incremental Learning}

Simple and well known regularization techniques are used with incremental learning strategies in order to control the tradeoff between old and new information. The regularization techniques that are used in the models are dropout, learning rate decay and layer freezing. Learning rate and its decay is used to control how large the weight updates are during the learning. A non-shared regularization mechanism is employed where every iteration has its hyperparameters re-initialized in order to make the model adaptive to most recent data. In other words while the weights of the network is transferred through iterations, the decayed learning rates are not.

In this case the assumption is a very small learning rate with high decay rate will make the model unresponsive to the latest iterations. Exponential learning rate decay (square root decay) is used in our experiments which means the learning rate will decrease through iterations. Dropout is used to prevent overfitting especially in incremental learning with data pooling where the model is updated for certain data points way more than the rest of dataset. Also a certain number of the layers are freezed so that corresponding layers' weights are not updated during training. The idea is to make sure that some of the previously learned knowledge is not overwritten by the incoming dataset. 

\section{Transfer Learning Strategies} \label{transfers}

During incremental learning, the model is trained at the end of each iteration with training data specified by the chosen incremental learning strategy. Since the model is trained continuously without reinitializing the weights in each iteration, this is equivalent to transfer learning between iterations. In the initial experimental setups, the training data comes from the same dataset divided into subsets provided as iterations. This experiment setup is expanded by transferring knowledge from a different paraphrasing dataset and start the incremental learning process with a pre-trained model or in other words with a knowledge base. 

Transfer learning from other sources are extremely relevant to the continuous human-in-the-loop setting especially in the cases where there are no resources to collect a large dataset. Since the training data is collected iteratively the model would have to work with even smaller dataset at the beginning. Transfer learning can help the model to learn better, providing a better initialization in the worst case scenario and increase the performance in this case. 

More importantly by training a model in a different dataset and using the pre-trained model to learn a new dataset, an artificial equivalent of concept drift is simulated no matter what transfer strategy is used. Basically in this case, there is a model which is used for a different dataset (different statistics, distribution, context etc.) than the one it is trained with. Only difference between this scenario and the concept drift scenario that exists in incremental learning is, in this case the model has access to all of the new dataset, instead of small chunks. Therefore finding good strategies for transfer learning can help with the concept drift problem if the transfer strategies is combined with incremental learning. 

Before starting with the transfer learning from other paraphrase datasets, experiments are conducted for answering these questions:

\begin{itemize}

\item Is knowledge transfer possible for neural paraphrase generation?

In order to answer this question, a model is trained with a large paraphrasing dataset as a source model. Then the transfer is done from this source model to target model with simplest way possible which is copying the weights of every component of source model and train the target model in a supervised manner. The performance is compared with traditional supervised learning without transfer to see if there is any improvement.

\item If answer to the first question is yes, what components of the model are being transferred?

To see what aspects of the source model is transferred, the experiments are run with three different settings; transfer embedding layer only, transfer embedding and hidden layers, transfer all layers including output layers. The performances are compared to see what components lead to improvement. 

\item Does transfer success depend on the characteristics of the participant datasets?

In order to speculate about what properties of the datasets relevant to knowledge transfer, experiments with three different source datasets and two target datasets, are run. The transfer performances are studied to establish some correlations between performance and properties like context, content, size of the datasets.

\end{itemize}

After establishing knowledge transfer is indeed possible, three different transfer strategies are proposed:

\begin{itemize}

\item INIT: Knowledge transfer is done by copying every component of source model to target model.  

\item Freeze n-layers: Same as INIT but first n layers of the model are freezed.

\item Surplus layer:  Same as INIT but a surplus layer is added before output layer and all the other layers are freezed.

\end{itemize}

Basically, proposed transfer strategies are designed for putting restrictions on the target model, directly effecting how much the model learns from training data. Whereas INIT puts no restrictions on the target model, surplus layer conservatively keeps the knowledge from source model and freeze n-layers approach is thought as a middle ground between them, making adjustments between knowledge learned from source dataset and target dataset . With transfer learning the model deals with a similar old vs. new knowledge tradeoff as incremental learning but with a different scale. In this case old knowledge part of the tradeoff is much larger and can have significant effects on the models performance. Moreover all of the transfer strategies proposed in this chapter are designed while specific use cases are considered. It is hypothesized that INIT is more suitable when the dataset for target model is large and the model has to be complex whereas surplus layer is more suitable when the dataset for target model is small and the model has to specialize (general to specific for the same domain). 

The same regularization techniques used for incremental learning strategies are used for transfer learning strategies as well, even though their effect on the performance can be not the same. Since it is essential to efficiently use the knowledge learned from the source dataset in the target dataset, learning rate for the target model becomes very important. A large learning rate might overwrite the existing weights very quickly which means losing the information gained before. Therefore in the experiments, a larger learning rate is used for source dataset and smaller learning rate is used for target dataset. The same exponential learning decay that is used in incremental learning, is kept.

\section{Active Learning Strategies} \label{actives}

Active learning is perfectly suitable for human-in-the-loop setting since it aims to enable learning with less amount of data. Not only it can reduce the cost of data acquisition, it can also enhance the overall process since it enables the model to learn faster which would mean earlier and more meaningful involvement with the users. As it is explained before the core idea behind active learning is determining the most difficult data points to paraphrase and make the users annotate those data points. The aim is to integrate active learning into the incremental learning with data pooling strategy and study its effects on the model's performance. It is hypothesized that since incremental learning with data pooling updates its weights towards to some data points in the dataset more than others, if those points are made sure to be most difficult data points to paraphrase, this can help the model to train better and faster.

It is hard to define a difficult or informative data point in case of paraphrase generation just because of the fact that it is a generation task not classification or regression. Contrary to other NLP tasks like paraphrase recognition or named entity recognition, in paraphrase generation the model only has the source texts to work with which limits the information it has on the dataset. For example in the case of paraphrase recognition, the dataset contains both of the paraphrase pairs and the annotator is only have to decide if they have the same meaning or not but nevertheless information from both of the sentences can be used for sampling. In this case, the sampling strategies which consist of heuristics regarding the training data, are based on only the source texts. Three different sampling strategies are combined with incremental learning with data pooling. Sampling techniques that are used for the experiments are:

\begin{itemize}

\item Random Sampling (RS): Sentences which are going to be paraphrased by the user are selected randomly. This is the default method for all of the other experiments in this thesis and used as a baseline.

\item N-gram Coverage (NC): Proposed in \cite{rubio} for interactive machine translation, this sampling technique focuses on selecting the rarest data points in terms of their n-grams. In other words, this technique aims to calculate how much new information a data point can provide. The hypothesis is rarest sentences have to be seen more for model to learn an accurate probability estimation. 

Before the training, number of each n-gram present in the training data is calculated and stored. Since human-in-the-loop approach is simulated, n-gram counts of the whole dataset can be calculated beforehand but in a real world application where new training data is constantly streamed, they should be updated after each iteration. An n-gram is labelled as rare when it appears less than A times in training data. Using this label, the score for a given source sentence f, is computed as:

\begin{equation}
C(f) = \frac{\sum_{n=1}^N \lvert{\nu^{<A}_{n}(f)} \lvert} {\sum_{n=1}^N \lvert{\nu_{n}(f)} \lvert} 
\end{equation}


where ${\nu_{n}(f)}$ is the set of n-grams with size n in f, ${\nu^{<A}_{n}}$ is the set of n-grams of size n in f that are rare and N is the maximum n-gram order. In experiments with incremental learning with data pooling, N = 4 is chosen as the maximum n-gram order and a value of 10 is chosen for the threshold A. In order to asses how well n-gram coverage represents the informativeness of data, an experiment is conducted with reverse n-gram sampling where the data points are chosen based on how frequent they are in the dataset in terms of n-grams. The score for reverse n-gram sampling is calculated as:

\begin{equation}
C(f) = \frac{\sum_{n=1}^N \lvert{\nu^{>=A}_{n}(f)} \lvert} {\sum_{n=1}^N \lvert{\nu_{n}(f)} \lvert} 
\end{equation}

where the terms are the same as n-gram sampling equation.

%\item Least Confidence (LC): A very intuitive sampling technique, it chooses data points in which the model has least confidence in. In the context of deep neural models this means to select the data points that are assigned with lowest probability according to the equation \cite{shen}:
%
%\begin{equation}
%1 - \max_{y_{1}.....y_{n}}  \mathbb{P}  \left[ {y_{1}.....y_{n}}  | \{ x_{ij} \}  \right]   
%\end{equation}
%
%For LSTM based sequence-to-sequence model used in this work, the score for a data point is calculated by using the probability of greedily decoded sequence.

\end{itemize}

Since MSR dataset contains a very small number of paraphrases, active learning strategies are only tried on QUORA dataset.




\chapter{Results}\label{results}

This chapter reports the results of experiments done with proposed learning strategies and discusses the findings. We start with evaluating each strategy separately followed by experiments which evaluates combinations of different strategies. We end with a overall discussion of the results.

\begin{table}[t]
\centering
\large
 \begin{tabular}{|c | c | c | c | c | c | c | c | c | c |} 
 \hline
 \% & 20 & 30 & 40 & 50 & 60 & 70 & 80 & 90 & 100 \\ [0.5ex] 
 \hline
  IL1 & 0.1 &  \textbf{0.92} &  \textbf{0.94} &  \textbf{1.56} &  \textbf{1.13} &  \textbf{1.11} &  \textbf{1.26} &  \textbf{1.37} &  \textbf{1.41}  \\ 
 \hline
  IL2 & 0.07 & 0.15 & 0.21 & 0.47 & 0.77 & 0.99 & 1.04 & 1.19 & 1.16 \\ 
 \hline
 Baseline & \textbf{0.12} & 0.27 & 0.6 & 1.32 & 0.96 & 1.08 & 1.24 & 1.26 & 1.34 \\ 
 \hline
 SL & - & - & - & - & - & - & - & - & 0.09  \\ 
 \hline
\end{tabular}
\caption{BLEU scores of incremental learning on MSR dataset. Each column represents an iteration, first row shows what percent of the dataset is used in that iteration. Each row except the last represents the performance of a particular learning strategy which are described in Chapter 3. The last column represents the results from \cite{brad}.}
\label{table:4.1}
\end{table}

\section{Neural Paraphrase Generation with Incremental Learning}

Table 4.1 presents the results of incremental learning strategies on MSR dataset. The model is highly sensitive to parameter tuning, failing to converge with high learning rates. We train the model on a validation set with different hyperparameter configurations, mainly with different dropout probabilities and learning rates, and select the best performing model. As it can be seen from the table, the model performs very poorly achieving a BLEU score of 1.34 with traditional supervised learning. We related this poor performance to dataset's small size and high complexity. Moreover we also report a result from [Brad et al.] in order to reinforce the difficulty of MSR dataset. Their model achieved a BLEU score of 0.09 on MSR dataset with supervised learning. 

Incremental learning with and without data pooling achieves BLEU scores of 1.41 and 1.16 respectively. We see that all of the learning strategies including the baseline, seem to improve over time increasing its performance in a steady fashion in every iteration even though there are anomalies in baseline and incremental learning with data pooling, 4th iteration with 50 percent of the data. It is also observed that when trained with data pooling in an incremental manner, the model performs slightly better than supervised learning. Nevertheless the model fails to generate meaningful and grammatically correct paraphrases in every case.

Even though we can see a clear pattern through iterations the results of this experiment is not conclusive due to model's very poor performance. Moreover the high variance between different runs highly suggest that the results are not meaningful and the neural network is basically modelling noise. Each learning strategy has been run 3 times and the variance of the best performing learning strategy which is the incremental learning with data pooling, is 0.617. This a very high number compared to actual BLEU scores achieved by the model. 

At the end we can easily conclude that generating coherent paraphrases for MSR dataset is impractical if it is not impossible for our model. Moreover it is not applicable to human-in-the-loop setting since it basically produces noise in early iterations. This severely undermines the involvement of model with data acquisition process in other words the models in unable to make meaningful suggestions to the user.

\begin{table}[b]
\centering
\large
\begin{adjustbox}{width=1.2\textwidth,center=\textwidth} 
 \begin{tabular}{|c | c | c | c | c | c | c | c | c | c |} 
 \hline
 \% & 20 & 30 & 40 & 50 & 60 & 70 & 80 & 90 & 100 \\ [0.5ex] 
 \hline
  IL1 & 10.49 &  \textbf{19.43} & \textbf{21.91} &  \textbf{23.30} &  \textbf{24.30} &  \textbf{24.50} &  \textbf{25.15} &  \textbf{25.45} &  \textbf{26.19}  \\ 
 \hline
  IL2 &  \textbf{10.68} & 17.58 & 18.79 & 19.06 & 19.50 & 19.46 & 19.35 & 19.54 & 19.70 \\ 
 \hline
 Baseline & 7.93 & 15.37 & 17.58 & 19.73 & 21.20 & 21.63 & 22.49 & 23.12 & 23.43 \\ 
 \hline
 SL & - & - & - & - & - & - & - & - & 22.90  \\ 
 \hline
\end{tabular}
\end{adjustbox}
\caption{BLEU scores of incremental learning on QUORA dataset. Each column represents an iteration, first row shows what percent of the dataset is used in that iteration. Each row except the last represents the performance of a particular learning strategy which are described in Chapter 3. The last column represents the results from \cite{Guptaetal}.}
\label{table:4.2}
\end{table}

Table 4.2 presents the results of incremental learning strategies on QUORA dataset. No hyperparameter tuning is done on the model during this experiment. As it can be seen from the table, in every case the model performs reasonably well at the end of last iteration. The baseline model with traditional supervised learning achieves a BLEU score of 23.43 and it is able to generate meaningful, grammatically correct paraphrases. We also report the results from \cite{Guptaetal}. Their model achieved a BLEU score of 22.9 in QUORA dataset with supervised learning. It is important to state that the training set used in this experiment is slightly larger than the reference work. Additionally a much larger test set is used in this work. Therefore it is not possible to make conclusive statements regarding the comparison of two models. Nevertheless the comparison at least shows that our model performs comparably to the other state-of-the-art approaches.

\begin{table}[t]
\centering
\large
\begin{adjustbox}{width=1.2\textwidth,center=\textwidth} 
 \begin{tabular}{|c | c | c | c | c | c | c | c | c | c |} 
 \hline
 \% & 20 & 30 & 40 & 50 & 60 & 70 & 80 & 90 & 100 \\ [0.5ex] 
 \hline
  0 & \textbf{10.68} & 19.13 & 20.45 & 20.83 & 20.44 & 20.47 & 20.78 & 20.31 & 21.11  \\ 
 \hline
  0.3 & 10.49 & \textbf{19.43} & \textbf{21.91} & \textbf{23.30} & \textbf{24.30} & \textbf{24.50} & \textbf{25.15} & \textbf{25.45} & \textbf{26.19} \\ 
 \hline
  0.5 & 8.48 & 17.05 & 19.62 & 21.58 & 22.93 & 23.26 & 23.92 & 24.58 & 25.09 \\ 
 \hline
\end{tabular}
\end{adjustbox}
\caption{BLEU scores of incremental learning with data pooling on QUORA dataset with different dropout probabilities.  Each column except the first represents an iteration, first row shows what percent of the dataset is used in that iteration. First column represents the dropout probabilities.}
\end{table}

Incremental learning with data pooling achieves a BLEU score of 26.19 at the last iteration, outperforming other strategies including supervised learning. Moreover it starts outperforming traditional supervised learning which uses all of the dataset, only with 60 percent of the dataset. The model starts adapting and improves its performance constantly through iteration. Even with using only half of the dataset, the model can provide reasonable paraphrases. 

The comparison between baseline incremental learning and incremental learning with data pooling indicates the reasons why data pooling performs better. From the perspective of individual iterations with both learning strategy, the model trains with same number of epochs and same amount of data. Only difference between to learning strategies is, in incremental learning with data pooling the model employs transfer learning inside the same dataset by not re-initializing its weight in every iteration contrary to baseline incremental learning. In other words the model that is trained in a particular iteration, is already fine tuned with respect to a specific portion of the dataset. The results show that we are able to utilize the information gained in previous iteration.

As stated in Chapter 3, the main concern about the incremental learning with data pooling strategy is overfitting to data points which are collected in earlier iterations. Considering how prone the neural models are for overfitting, this would be an expected situation but the results show that the model does not overfit when it is trained with incrementally with data pooling. Training error of the model decreases and stabilizes through epochs. We hypothesize that the regularization methods we use, are preventing the model to overfit and making the learning process stable. In order to confirm this we experiment with different dropout probabilities and report the model's behaviour in incremental learning setting. Table 4.3 shows the effect of different dropout probabilities on incremental learning with data pooling.

As it can be seen from the table, the model is able to learn with the dropout probabilities of 0.3 and 0.5 while it fails to adapt when there is no dropout. Even though training error decreases steadily in every iteration, the model can not improve on the test set, overfitting over time which is the expected behaviour. We also train the incremental learning baseline with different dropout probabilities. The results are shown in Table 4.4. As it can be seen from the table, the model does not overfit even though no dropout is used.

\begin{table}[t]
\centering
\large
\begin{adjustbox}{width=1.2\textwidth,center=\textwidth} 
 \begin{tabular}{|c | c | c | c | c | c | c | c | c | c |} 
 \hline
 \% & 20 & 30 & 40 & 50 & 60 & 70 & 80 & 90 & 100 \\ [0.5ex] 
 \hline
  0 & \textbf{10.59} & 15.27 & 19.20 & 19.69 & 21.60 & 18.69 & 22.37 & 22.86 & 23.09  \\ 
 \hline
  0.3 & 7.93 & \textbf{15.37} & \textbf{17.58} & \textbf{19.73} & \textbf{21.20} & \textbf{21.63} & \textbf{22.49} & \textbf{23.12} & \textbf{23.43} \\ 
 \hline
  0.5 & 8.19 & 13.34 & 15.33 & 17.09 & 18.36 & 19.05 & 20.09 & 20.94 & 21.54 \\ 
 \hline
\end{tabular}
\end{adjustbox}
\caption{BLEU scores of incremental learning baseline on QUORA dataset with different dropout probabilities. Each column except the first represents an iteration, first row shows what percent of the dataset is used in that iteration. First column represents the dropout probabilities.}
\end{table}

Incremental learning without data pooling achieves a BLEU score of 19.70 at the last iteration. The results show that even though the model improves its BLEU score in first 4 iterations it fails to adapt afterwards achieving similar BLEU scores for last 5 iterations. The model has the highest score in last iteration like the other learning strategies but increase in performance is not notable. Moreover at the last iteration, incremental learning without data pooling performs worse than both incremental learning baseline and incremental learning with data pooling.

\begin{table}[b]
\centering
\large
\begin{adjustbox}{width=1.2\textwidth,center=\textwidth} 
 \begin{tabular}{|c | c | c | c | c | c | c | c | c | c |} 
 \hline
 \% & 20 & 30 & 40 & 50 & 60 & 70 & 80 & 90 & 100 \\ [0.5ex] 
 \hline
  5 & 10.68 & 17.58 & 18.79 & \textbf{19.06} & \textbf{19.50} & \textbf{19.46} & \textbf{19.35} & \textbf{19.54} & \textbf{19.70}  \\ 
 \hline
  10 & 17.07 & \textbf{18.62} & \textbf{18.83} & 18.54 & 18.76 & 19.12 & 19.15 & 19.45 & 19.69 \\ 
 \hline
 15 & \textbf{18.96} & 18.07 & 18.47 & 18.46 & 18.87 & 19.05 & 19.11 & 19.15 & 19.19 \\ 
 \hline
\end{tabular}
\end{adjustbox}
\caption{BLEU scores of incremental learning without data pooling on QUORA dataset with different number of epochs. Each column except the first represents an iteration, first row shows what percent of the dataset is used in that iteration. First column represents the number of epochs.}
\end{table}

When compared to baseline incremental learning, it is easy to see that at the last iteration both models are trained the same number of epochs. The baseline model is outperformed by the model trained with incremental learning without data pooling in the first 3 iterations but starting with the 4th iteration it begins to outperform the other. This is an expected result because the continuous model has better weight initialization when there is not enough training data for the model to learn properly.

Both of the incremental learning strategies we propose uses transfer learning in the same dataset only difference being the resources they transfer from. Incremental learning with data pooling not only uses more training data to transfer between iterations, it also updates certain parts of the datasets more. It can be argued that the reason behind the low performance of incremental learning without data pooling is the fact that it is not updating the models weights enough hence underfitting to the whole dataset. We experiment with different number of epochs per iteration in order to test this theory. The results are shown in Table 4.5.

As it can be seen from the table, the number of epochs used in training does not effect the performance significantly. The model trained with number of epochs of 5, 10 and 15, scored BLEU scores of 19.70, 19.69 and 19.82 respectively at the last iteration. Training error analysis of the model clearly shows that the model is able to learn pretty fast since the training error continuously decreasing and converges in every case. This might be attributed to size of the training data in each iteration considering the fact that the model can fit itself to the small dataset faster.

Since the model is not underfitting when it is trained with training data from separate iterations, low perplexity of the training clearly shows this, it would be insightful to look at other aspects of the model's behaviour. Another explanation of the low performance of incremental learning without data pooling could be ineffectiveness of knowledge transfer from the same dataset. It is important to point out that the notion of low performance indicates the model's inability to adapt rather than the BLEU score at the last iteration. There is a possibility that the model is completely unable to transfer the knowledge it gained from the previous iterations. In order to test this hypothesis, we train the model in a supervised manner without data pooling and with re-initializing (no transfer between iterations). Table 4.6 shows the results. 

\begin{table}[t]
\centering
\large
 \begin{tabular}{|c | c | c | c | c | c | c | c | c | c |} 
 \hline
 \% & 20 & 30 & 40 & 50 & 60 & 70 & 80 & 90 & 100 \\ [0.5ex] 
 \hline
  BLEU & 10.15 & 12.87 & 12.65 & 16.03 & 15.82 & 16.19 & 16.62 & 16.77 & 16.35  \\ 
 \hline
\end{tabular}
\caption{BLEU scores of supervised learning without data pooling on QUORA dataset. Each column represents an iteration, first row shows how much percent of the dataset is used in that iteration. Each iteration trained separately.}
\end{table}

As it can be seen from the table, incremental learning performs better than supervised learning in every iteration. This shows that continually training the model and transferring the weights through iterations has an effect on model's performance. It can be easily seen that the model is able to use some part of previous knowledge. From the analysis on incremental learning without data pooling, we hypothesize that even though the model is able to utilize past information, it gradually forgets it. In other words the model overwrites the weights which are fine tuned in previous iterations.

\section{Neural Paraphrase Generation with Transfer Learning}

\begin{table}[t]
\centering
\large
 \begin{tabular}{|c | c | c |} 
 \hline
 Source Dataset & QUORA & PPDB \\ [0.5ex] 
 \hline
  INIT & 8.9 & \textbf{3.08}  \\ 
 \hline
  Freeze 1 layer & \textbf{10.04} & 1.73  \\ 
 \hline
  Freeze 2 layers & 8.74 & 2.46  \\ 
 \hline
  Surplus layer & 8.32 & 2.32  \\ 
 \hline
  Embedding only & 2.38 & 0.73  \\ 
 \hline
  Supervised Learning & 1.34 & 1.34  \\ 
 \hline
\end{tabular}
\caption{BLEU scores of different transfer learning methods on MSR dataset. First row represents the source datasets. Rest of the rows represents transfer methods and corresponding BLEU scores.}
\end{table}

Table 4.7 presents the results of transfer learning on MSR dataset. QUORA and PPDB-Lexical datasets are used as source datasets. Both source datasets are trained for 10 epochs and with same hyperparameter configuration. No additional parameter tuning is been done for neither of the settings.

As it can be seen from the table, transfer learning improves BLEU score with every transfer method except one for both source datasets. When transferred from QUORA dataset, the model improves its BLEU score by 9.7 at the best case which is a very significant improvement since the original model could not produce acceptable paraphrases at all. When transferred from PPDB even though the model is still unable to generate acceptable paraphrases, it improves its BLEU score only by 1.74 at the best case which is an unexpected result. Our results on transfer learning from PPDB-Lexical dataset do not coincide with the results of [Bali et al] where they report a BLEU score of 10.29 for transfer from PPDB-Lexical to MSR. One possible explanation for this inconsistency would be the lack of parameter tuning for source datasets in our experiments. Since the source model for PPDB-Lexical is not fine tuned, it might not perform well in MSR. Another explanation could be the difference in models used for paraphrase generation task but we find this very unlikely since our model is able to generate results that are comparable to state-of-the-art architectures in though it is a fairly simple model. In retrospect, the result that using QUORA dataset as source leading to better performance than using PPDB-Lexical, is not suprising. Even though PPDB-Lexical dataset is larger, it contains no context information unlike QUORA dataset. Moreover QUORA and MSR datasets have similar domains (more common words and topics, etc.) which should lead to better performance.

The transfer method which performs best for QUORA dataset is freeze 1 layer. When QUORA dataset used as source, rest of the transfer methods performs similarly, INIT improving the BLEU score by 6.52, freeze 2 layers improving by 6.36 and surplus layer by 5.98. For PPDB dataset, the best transfer method is INIT. Freeze 1 layer improves the BLEU score by 1, freeze 2 layers by 1.73 and surplus layer by 1.59. When only the trained word embeddings are transferred the model improves by 1.04 with QUORA dataset as source. This result shows that hidden layers are not the only aspect of the model that is transferable. When PPDB-Lexical dataset is used as source, transferring the word embeddings actually hurts the performance. This result is quite inconsistent with other transfer learning experiments and it lacks an explanation.

\begin{table}[t]
\centering
\large
 \begin{tabular}{|c | c | c |} 
 \hline
 Target Dataset & QUORA-120K & QUORA-24K \\ [0.5ex] 
 \hline
  INIT & 22.71 & 13.46  \\ 
 \hline
  Freeze 1 layer & 21.73 & \textbf{14.80}  \\ 
 \hline
  Freeze 2 layers & 21.56 & 12.93  \\ 
 \hline
  Surplus layer & 19.37 & 12.08  \\ 
 \hline
  Embedding only & 21.89 & 12.57  \\ 
 \hline
  Supervised Learning & \textbf{23.43} & 10.49  \\ 
 \hline
\end{tabular}
\caption{BLEU scores of different transfer learning methods on QUORA-120K and QUORA-48K datasets when MSCOCO dataset used as source. First row represents the target datasets. Rest of the rows represents transfer methods and corresponding BLEU scores.}
\end{table}

Table 4.8 presents the results of transfer learning on QUORA dataset from MSCOCO dataset. Pre-trained source model shares the same hyperparameter configuration with previous source models and it is trained for 10 epochs. The results show that transferring MSCOCO to QUORA dataset is not effective moreover it decreases the performance. All of the transfer methods performs worse than traditional supervised learning without transfer. Among the transfer methods only INIT has a comparable performance to supervised learning with a BLEU score of 22.21 whereas supervised learning achieves a BLEU score of 22.40. This is an unexpected result since it suggest that MSCOCO dataset has no useful knowledge which can help with the training of QUORA dataset. It can be argued that this is a similar problem to the knowledge forgetting problem which is observed with incremental learning without data pooling. We hypothesize that since there is enough training data to learn from the transferred knowledge is overwritten by the target task. In order to investigate this behaviour better, we randomly select a subset of QUORA dataset with the size of 24,000 paraphrase pairs and try knowledge transfer from MSCOCO to this subset. The original dataset contains approximately 120,000 paraphrase pairs. Results show that transfer learning performs better in this case, improving the BLEU score by 4.31. All of the transfer methods improve the model's final score and freeze 1 layer approach performs best.

Transferring output layers makes no significant improvements to the performance for both QUORA and MSR target datasets.


\section{Neural Paraphrase Generation with Active Learning}


\section{Neural Paraphrase Generation with Network Expansion}

Table 4.9 presents the results of incremental learning with network expansion on QUORA dataset. As it can be seen from the table, incremental learning with and without data pooling achieve BLEU scores of 21.26 and 14.77 at the last iteration.

In the case of incremental learning with data pooling, the model gradually increases its performance except when a new layer is added (iterations 4 and 7). On the other hand, in the case of Incremental learning without data pooling, the model improves until the first extra layer is added (iteration 4).  After that point on adaptiveness of the model diminishes almost completely.

When the model's layers are not freezed with network expansion, in incremental learning without data pooling, the model's behaviour does not change much, meaning it still loses its adaptiveness while its overall performance increases by 2.23.

\begin{table}[t]
\centering
\large
\begin{adjustbox}{width=1.2\textwidth,center=\textwidth} 
 \begin{tabular}{|c | c | c | c | c | c | c | c | c | c |} 
 \hline
 \% & 20 & 30 & 40 & 50 & 60 & 70 & 80 & 90 & 100 \\ [0.5ex] 
 \hline
  IL1-NE & \textbf{10.89} & 18.07 & 19.86 & 19.07 & 20.09 & 20.49 & 19.78 & 20.95 & 21.26  \\ 
 \hline
  IL2-NE & 10.14 & 15.86 & 16.77 & 15.00 & 15.55 & 15.70 & 15.03 & 15.42 & 14.77 \\ 
 \hline
  IL2-NE2 & 9.45 & 16.24 & 18.05 & 16.81 & 17.33 & 17.83 & 16.75 & 17.05 & 17.00 \\ 
 \hline
  IL1 & 10.49 &  \textbf{19.43} & \textbf{21.91} &  \textbf{23.30} &  \textbf{24.30} &  \textbf{24.50} &  \textbf{25.15} &  \textbf{25.45} &  \textbf{26.19}  \\ 
 \hline
  IL2 &  10.68 & 17.58 & 18.79 & 19.06 & 19.50 & 19.46 & 19.35 & 19.54 & 19.70 \\ 
 \hline
\end{tabular}
\end{adjustbox}
\caption{BLEU scores of incremental learning with and without network expansion on QUORA dataset. Each column represents an iteration, first row shows what percent of the dataset is used in that iteration. IL2-NE2 represents incremental learning without data pooling and layer freezing.}
\end{table}

\section{Combining Different Strategies}

Results of the experiments done with MSR dataset shows that the model fails to learn the dataset properly with incremental or supervised learning but the model is able to generate acceptable sentences while transfer learning is used. Since it is not possible to analyse the performance of incremental learning on MSR dataset as it is, we train the model with both transfer and incremental learning. A source model is trained on another dataset and used as a knowledge base for target model. Target model is then trained incrementally using different transfer methods. With this experiment setup we also simulate a concept drift situation since a model that is trained on one particular dataset, is getting exposed to a different dataset. Table 4.10 presents the results of incremental learning with data pooling with different transfer methods. QUORA dataset is used as source dataset in this experiment and for target model, we keep the same hyperparameter configuration which is determined by using the validation set, used in previous experiments.

\begin{table}[t]
\centering
\large
\begin{adjustbox}{width=1.2\textwidth,center=\textwidth} 
 \begin{tabular}{|c | c | c | c | c | c | c | c | c | c |} 
 \hline
 \% & 20 & 30 & 40 & 50 & 60 & 70 & 80 & 90 & 100 \\ [0.5ex] 
 \hline
  INIT & \textbf{5.23} & \textbf{6.28} & 6.90 & \textbf{8.31} & \textbf{8.97} & 8.93 & \textbf{10.39} & 10.70 & \textbf{11.20}  \\ 
 \hline
 Freeze 1 layer & 4.23 & 5.99 & 6.74 & 7.77 & 8.71 & \textbf{9.09} & 9.29 & 10.14 & 10.37 \\ 
 \hline
 Freeze 2 layers & 5.06 & 5.94 & 7.36 & 7.74 & 8.88 & 8.86 & 10.15 & \textbf{10.77} & 11.03 \\ 
  \hline
  Surplus layer & 1.44 & 3.61 & 3.86 & 4.74 & 4.89 & 5.37 & 5.12 & 6.12 & 6.57 \\ 
  \hline
  SL & - & - & - & - & - & - & - & - & 10.04 \\ 
 \hline
\end{tabular}
\end{adjustbox}
\caption{BLEU scores of incremental learning with data pooling trained with different transfer methods on MSR dataset. Each column except the first represents an iteration, first row shows what percent of the dataset is used in that iteration. Last row represents supervised learning.}
\end{table}

As it can be seen from the table, all the transfer methods steadily improve their performances through iterations which is an expected behaviour from previous experiments. Every transfer method except the surplus layer outperforms traditional supervised learning. Best performing transfer method is INIT with a BLEU score of 11.20 whereas supervised learning achieves 10.04. Freezing the first layer and first two layers achieves BLEU scores of 10.37 and 11.03 respectively. Additionally, the best transfer method outperforms supervised learning with only 80 percent of the dataset which is a results that is consistent with previous experiments.

Transferring by adding a surplus layer and freezing the rest does not work well, achieving a BLEU score of 6.57 underperforming even the supervised learning. Training error analysis of the method clearly shows why it is not performing well. Training error of the model with surplus layer stabilizes much slower and at a much larger value than other transfer methods. Moreover, it is observed that the paraphrases that are generated by the model during training, are dominated by the source dataset. In other words the model dominantly chooses words from the source dataset. At the end it is clear that the model underfits to MSR dataset when it transfers with a surplus layer.

\begin{table}[t]
\centering
\large
 \begin{tabular}{|c | c | c | c | c | c | c | c | c | c |} 
 \hline
 \% & 20 & 30 & 40 & 50 & 60 & 70 & 80 & 90 & 100 \\ [0.5ex] 
 \hline
  IL1 & \textbf{5.23} & \textbf{6.28} & \textbf{6.90} & \textbf{8.31} & \textbf{8.97} & \textbf{8.93} & \textbf{10.39} & \textbf{10.70} & \textbf{11.20}  \\ 
 \hline
  IL2 & 5.05 & 3.36 & 3.96 & 4.35 & 4.32 & 3.93 & 4.29 & 4.27 & 4.274 \\ 
 \hline
  IL3 & 5.09 & 5.90 & 6.43 & 7.57 & 8.52 & 8.79 & 9.86 & 9.98 & 10.02 \\ 
  \hline
  SL & - & - & - & - & - & - & - & - & 10.04 \\ 
 \hline
\end{tabular}
\caption{BLEU scores of different incremental learning strategies trained with INIT transfer method on MSR dataset. Each column except the first represents an iteration, first row shows what percent of the dataset is used in that iteration. Last row represents supervised learning.}
\end{table}

Table 4.11 presents the results of transfer method INIT with different incremental learning strategies. QUORA dataset is used as source dataset in this experiment as well. 

As it can be seen from the table while incremental learning with data pooling and baseline incremental learning improves their performance through iterations, incremental learning without data pooling is unable to adapt. Results of incremental learning on MSR dataset is quite consistent with the results on QUORA dataset. Model's behaviour with different incremental learning strategies are similar in both datasets. Only notable difference between two sets of experiments is the underfitting in incremental learning without data pooling on MSR dataset. The model is unable to learn well on training data without pooiling, producing large training errors during training. We hypothesize that reason for this is small size of the dataset and there is simply not enough data in one iteration for model to learn properly. 

\begin{table}[b]
\centering
\large
\begin{adjustbox}{width=1.2\textwidth,center=\textwidth} 
 \begin{tabular}{|c | c | c | c | c | c | c | c | c | c |} 
 \hline
 \% & 20 & 30 & 40 & 50 & 60 & 70 & 80 & 90 & 100 \\ [0.5ex] 
 \hline
  INIT & 13.64 & 17.03 & 20.10 & 21.14 & 22.34 & 22.78 & 23.77 & 24.41 & 24.85  \\ 
 \hline
  Freeze 1 layer &  \textbf{14.80} & 17.58 & 19.33 & 20.68 & 21.87 & 22.19 & 23.52 & 23.80 & 24.35 \\ 
  \hline
  IL1 & 10.49 &  \textbf{19.43} & \textbf{21.91} &  \textbf{23.30} &  \textbf{24.30} &  \textbf{24.50} &  \textbf{25.15} &  \textbf{25.45} &  \textbf{26.19}  \\ 
 \hline
\end{tabular}
\end{adjustbox}
\caption{BLEU scores of incremental learning with data pooling trained with transfer methods INIT and Freeze 1 layer on QUORA dataset. Each column except the first represents an iteration, first row shows what percent of the dataset is used in that iteration. Last row represents incremental learning without transfer.}
\end{table}

Table 4.12 presents the results of incremental learning with data pooling with INIT and freeze 1 layer transfer methods. MSCOCO dataset is used as source and QUORA dataset is used as target. We keep the same hyperparameter configuration as previous experiments since no parameter tuning is done for QUORA dataset.

As it can be seen from the table the model is able to gradually increase its performance in both cases. At the last iteration, model achieves BLEU scores of 24.85 and 24.35 using transfer methods INIT and freeze 1 layer respectively. Results are consistent with previously reported transfer learning experiments where the model without transfer is outperforming the model with transfer. As reported in section 4.1.1, incremental learning with data pooling achieves a BLEU score of 26.19 without any transfer.  

\section{Discussion}

All the experiments which are conducted regarding incremental learning, show that incremental learning with data pooling has the best performance. Even though the baseline incremental learning (which is basically supervised learning) also increase its performance through time, the difference between BLEU scores clearly shows that continuously training the model leads to a better tuned and more data efficient model. On both datasets, incremental learning with data pooling is able to outperform traditional supervised learning without seeing 100 percent of the dataset. It is also important to notice the fact that baseline incremental learning and incremental learning with data pooling trains for same number of epochs. Therefore their total training times are approximately the same. 

Before the experiments main concern regarding to incremental learning was overfitting but the results show that overfitting is avoided because of the regularization mechanisms. Especially dropout mechanism effectively prevents overfitting and makes incremental learning with data pooling possible. Once overfitting is removed from the equation (of course it is impossible to completely eliminate overfitting), it is not hard see why incremental learning with data pooling works better. By continuously training through iterations, the model starts the next iterations with better weight initializations and previous knowledge regarding the dataset. In other words we apply transfer learning between the same dataset (this is also correct for incremental learning without data pooling). Positive effect of incremental learning is evident if learning curve of the model is analyzed. During the experiments we observed that for each iteration the training and validation errors in the beginning are lower than previous iterations. Another interesting aspect of this learning strategy is the amount of increase in BLEU score through iterations. Rate of performance gain is higher in the early iterations than late iterations on both datasets. One would expect similar increases in BLEU scores in MSR dataset since it is a very hard dataset to paraphrase. Training and validation errors are much higher on MSR than on QUORA therefore there should be more room for the model to learn in MSR (stabilizing in later iterations).

Incremental learning without data pooling does not perform well regarding both final BLEU score and adaptivity which is an expected outcome. The behaviour of this learning strategy was similar in all experiments (on both datasets, with network expansion, transfer learning and different configurations). The model starts improving in first half of the iterations and adaptivity severely diminishes after that point. The interesting finding is the fact that model learns separate iterations rather quickly but fails to improve its performance in test set which suggests catastrophic forgetting between iterations. In the end incremental learning without data pooling does not seem to be a suitable training strategy for paraphrase generation.

With our findings on incremental learning, we can conclude that the answer to our first research question, "\textit{Can deep neural models effectively adapt and gradually perform in a continuous data stream for paraphrase generation?}", is yes. We show that with incremental learning baseline and incremental learning with data pooling, we can successfully integrate a deep neural model into human-in-the-loop setting for paraphrase generation. We demonstrate the adaptivity of incremental learning with data pooling and its superiority over supervised learning and baseline incremental learning, in this particular setting. Moreover we can also partially answer our second research question "\textit{Can we achieve better or comparable performance than traditional supervised learning by leveraging the data stream?}" since we successfully show that incremental learning learns better and faster on two different paraphrase datasets.


Results of experiments which are done regarding transfer learning is not conclusive. While transfer learning improves BLEU score for MSR dataset significantly in some cases, it does not work well for QUORA dataset even decreasing the performance. In the case of MSR, one of our findings are inconsistent with \cite{brad} regarding the effect of transfer from PPDB-Lexical dataset. While they report significant improvements, our experiments showed little to none improvement and even decrease in performance when we only transfer word embeddings. One can argue that the reason for our findings is the poor performance of source model since we do not do parameter tuning on source models, but the difference between reported BLEU score is too high which makes this explanation highly unlikely. Our initial expectation of high performance from QUORA as a source dataset, is confirmed to be right. We assumed that the transfer would be successful since the two datasets are similar (similar domain, similar words etc.) and transfer from QUORA resulted in a significant improvement in BLEU score. In the case of QUORA dataset, transfer from MSCOCO only works if we have a small subset of the dataset. We conclude that when the model has access to enough data, transferred knowledge becomes irrelevant pretty quickly even though MSCOCO dataset is almost twice as big as QUORA. We also see that when transfer learning works, hidden states (network weights) are not the only aspect of source model that is being transferred. Word embeddings are also transferrable. 

Regarding to different methods, applying restrictions to the model during transfer does not work as intented. The transfer is usually more successful when the model is less restricted but performances of four different transfer methods are close to each other. Our findings agrees with \cite{mou}, shows that transfer learning for NLP tasks highly depends on the task, model and datasets. Since we are able to create a model that is impossible to get created without transfer, our findings on transfer learning can partially answer our third research question, "\textit{What are the possible challenges/limitations introduced by system and model in this setting and what are the possible learning strategies we can use for dealing with these challenges?}". In a situation in which there is a very small and challenging dataset and no way to obtain more data, transfer learning can be the solution depending on the task and the datasets.

Most interesting finding regarding transfer learning is; when incremental learning is combined with transfer learning, the idea of specialization does not work at all for LSTM based models for paraphrase generation. In our experiments we have seen that the more transfer method restricts model's training the worse the performance when the model is continuously trained. Since QUORA dataset is way larger than MSR dataset, adding a new layer to the model and freezing rest of the network seemed to be a good idea since we have a large and fine tuned general model to transfer from. Making only one layer to specialize on MSR dataset looks like a natural option but this transfer methods fails miserably by underfitting to MSR dataset. If the transfer is possible for supervised learning, contrary to surplus layer other transfer methods works well when they are trained with incremental learning with data pooling, learning faster and better than traditional supervised learning. Therefore our findings on incremental transfer learning completes the answer of our second research question.


Basic idea behind network expansion during incremental learning was to introduce some flexibility to model and slow down the forgetting behaviour observed during training. This does not work as intented since expanding the network had no positive effect on neither of the incremental learning strategies. Expanding the network without freezing the other layers shows some promising results and needs investigating.







\chapter{Conclusion}\label{conclusion}

This thesis studies neural paraphrase generation in settings where the training data is collected from a continuous data stream. A continuous, human-in-the-loop data acquisition process is simulated with existing datasets. A data driven and adaptive learning paradigm is proposed for training the model. A deep neural model based on stacked LSTM's is implemented and its behaviour is analyzed under different learning strategies. It is shown that integration of the model with the data stream is possible and it improves the performance of paraphrase generation task. Since the research on adaptive neural models for NLP tasks is quite limited, actual applicability of the concept is the main focus of this thesis. The work in this thesis can be considered as a proof of concept regarding adaptive models in continuous, human-in-the-loop settings.

Three different incremental learning strategies (including an incremental learning baseline to make meaningful comparisons) are proposed and experimented with, in order to show the model's ability to work in human-in-the-loop settings. Experiments with two different datasets have been conducted using iterative evaluation to monitor the learning process. Although one of the incremental learning strategies underperforms and fails to adapt through time, the other strategy is shown to be quite successful regarding both adaptivity and overall performance. Experiments also demonstrate that the successful incremental learning strategy outperforms the baseline with less amount of data. At the end, the practicality of adaptive learning over supervised learning is clearly demonstrated by the experiments.

For the cases where large amount of training data is not available, the effect of transfer learning on paraphrase generation is investigated. The possibility of knowledge transfer for neural paraphrase generation is questioned and an analysis is done in order to determine what aspects of the network are being transferred. It is shown that for paraphrase generation, the transfer is successful when the target dataset is low resource. It has a negative effect on performance when the target dataset is large. It is also shown that while the transfer of the output layers does not make any difference, both hidden layer weights and word embedding weights have an impact on performance during transfer. Additionally, four different transfer methods are proposed to be employed in different specific use cases. Initial hypothesises which are considered during the design of transfer methods fail since the experiments show that restricting the model's learning ability during transfer leads to poor performance.

In order to utilize the human-in-the-loop setting better by selecting the best samples to be annotated, three different active learning strategies for are proposed. It is clear from the experimental results that n-gram coverage of a sentence does not represent the informativeness of that sentence in paraphrase generation.

Finally the proposed learning strategies are tested in a setting where a concept drift occurs. A previously trained model encounters new training data which does not come from the original dataset the model trained with. Proposed learning strategies are combined in order to deal with this problem. The results show that in the case of a concept drift, the model is able to adapt to the new training data with incremental learning with data pooling. 



\chapter{Future Work}\label{conclusion}

There are many research questions regarding adaptive neural models for NLP which are not addressed in this thesis due to the time limitations. This section gives an overview of the future work which can address the remaining research questions. 

The experiments that are done in the thesis only simulates a human-in-the-loop data acquisition process. Since the only part that can be simulated is the iterative, continuous nature of the process, the actual human input on model's predictions are not utilized in the experiments. First and most important future work would be integrating the neural model to a real word human-in-the-loop data acquisition tool like \cite{par4sim}. It is important to study the impact of interactive human evaluation and correction on the model. 

Parallel to the first future work, a human evaluation of the paraphrases that are generated by the model, is also important. Considering the fact the BLEU is an automatic metric which is not designed for paraphrase generation, incorporating the human evaluation of the suggested paraphrases could be helpful to decide when to stop the data acquisition. Extending the user interface to provide exactly that would be an improvement to the data acquisition tool.

As it is stated in the first chapter, when paraphrasing source sentences, the users choose from a list of paraphrase suggestions created by the model and external resources. With this way it is ensured that the paraphrases created by human users come from a vocabulary which is known by the model. Even though it introduces restrictions to the users, this has to be done since the model cannot change it's vocabulary after the training is started. To tackle this problem, very large pre-trained word embeddings like GloVe \cite{glove} or Word2Vec Google News \cite{mikolov} can be used. This can improve the performance and make the model more robust.

Regardless the way of how the vocabulary is created, there will always be some words that are not in the vocabulary. Especially in a setting where the training data is constantly streamed into the model, out of the vocabulary words are inevitable. A mechanism to intelligently handle out of vocabulary words, like \cite{ofw} , should be integrated into the whole data acquisition process.

Transferring from different datasets in the same domain is not always effective especially if the target dataset is large enough. This assertion can be false when the transfer learning is done from different tasks as it is demonstrated in \cite{brad}. As future work the effect of transferring from different tasks for example entailment generation, can be investigated. This might be an important improvement on the current framework since it might be not possible to find a different dataset in the same domain if the domain is very specific.

Finally, since the research on active learning for paraphrase generation is non existent it is a natural future work for this thesis. The sampling methods which are experimented on in this thesis are shown to be unsuccessful therefore new sampling methods for neural paraphrase generation is still an open research question.

\cleardoublepage


%%%%%%%%%%%%%%%%%%%%%%%%%%%%%
%% Appendices:
%% these are optional! For most Bachelor-theses and some Master-thesis none of them is needed. 
%% Just comment them if not needed.
%\appendix
%\fancyhead[LO,RE]{}                      % Define the header style for the appendixpages
%
%\fancyhead[LE,RO]{\it Anhang A. Nomenclature}                %Adapt letter!
%  \input{appA.tex}
%\cleardoublepage
%
%\fancyhead[LE,RO]{\it Anhang B. Additional Proofs}            %Adapt letter!
%  \input{appB.tex}
%\cleardoublepage
%
%\fancyhead[LE,RO]{\it Anhang C. Complete Simulation Results}%Adapt letter!
%  \input{appC.tex}
%\cleardoublepage

% ... add as much appendices as you need (one can also add source code, for example)

%\fancyhead[LE]{\it \leftmark}
%\chapter{}
\fancyhead[LE,RO]{\it Bibliography}       % A bibliography never have a letter or numbering!
    \bibliographystyle{apalike}             % Style for presenting the literature
	\phantomsection\addcontentsline{toc}{chapter}{Bibliography}% Add to the TOC
    \bibliography{thesis}
%\cleardoublepage

%%%%%%%%%%%%%%%%%%%%%%%%%%%%%
% Formal page 1
\vspace{2cm}
\chapter*{Erkl\"arung der Urheberschaft}
\label{sec:urheber}
\fancyhead[LE]{\it Erkl\"arung der Urheberschaft}
	Hiermit versichere ich an Eides statt, dass ich die vorliegende
	\trtype{} im Studiengang \trcourseofstudies{}
	selbstst{\"a}ndig verfasst und keine anderen als die angegebenen
	Hilfsmittel - insbesondere keine im Quellenverzeichnis nicht
	benannten Internet-Quellen – benutzt habe. Alle Stellen, die
	w{\"o}rtlich oder sinngem{\"a}{\ss} aus Ver{\"o}ffentlichungen entnommen wurden,
	sind als solche kenntlich gemacht. Ich versichere weiterhin, dass
	ich die Arbeit vorher nicht in einem anderen Pr{\"u}fungsverfahren
	eingereicht habe und die eingereichte schriftliche Fassung der
	auf dem elektronischen Speichermedium entspricht.
	
\vspace{4cm}
\noindent Ort, Datum \hfill Unterschrift

%The backcover is always empty
\newpage
\thispagestyle{empty}
\hspace{1cm}
\newpage

%%%%%%%%%%%%%%%%%%%%%%%%%%%%
% Formal page 2
\vspace{2cm}
\chapter*{Erkl\"arung zur Ver\"offentlichung}
\label{sec:urheber}
\fancyhead[LE]{\it Erkl\"arung zur Ver\"offentlichung}
Ich stimme der Einstellung der \trtype{} in die Bibliothek des Fachbereichs Informatik zu.

\vspace{4cm}
\noindent Ort, Datum \hfill Unterschrift

%The backcover is always empty
\newpage
\thispagestyle{empty}
\hspace{1cm}
\newpage

\end{document}
%EOF

